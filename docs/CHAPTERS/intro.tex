\section{Introduction}

The \textbf{F-LOW Food Delivery Service} is a fictional software system developed as an assignment for the \textit{COMP.SE.210 Software Systems Architecture} course at Tampere University. The system seeks to automate food delivery services by connecting restaurants, customers, and couriers onto one platform.

This project emphasizes architectural modeling and using the \textit{4+1 view model} to describe the system from multiple perspectives. In this middle phase, we focus on the capture of the functional requirements of the system through a comprehensive use case diagram and corresponding scenario descriptions. These artifacts form the foundation for architectural decisions and guarantee that stakeholder requirements are met in an organized and traceable manner.

The project has been developed using git as VCS. This document has been formatted using LaTex and all diagrams from it have been exported from Papyrus.

Word Count: 
